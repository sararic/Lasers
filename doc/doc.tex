\documentclass[12pt]{article}

\usepackage[utf8]{inputenc}
\usepackage[T1]{fontenc}
\usepackage{amsfonts}
\usepackage{textgreek}
\usepackage{appendix}
\usepackage{subfiles}
\usepackage{amsmath}
\usepackage{amssymb}
\usepackage{ccaption}
\usepackage{caption}
\usepackage{chngcntr}
\usepackage{enumitem}
\usepackage{fancyhdr}
\usepackage{float}
\usepackage{gensymb}
\usepackage{geometry}
\usepackage{graphicx}
\usepackage[colorlinks=true, linkcolor=blue, urlcolor=blue, citecolor=red]{hyperref}
\usepackage{lastpage}
\usepackage{layout}
\usepackage{multicol}
\usepackage{multirow}
\usepackage[final]{pdfpages} 
\usepackage{pstricks}
\usepackage{siunitx}
\usepackage{soul}
\usepackage{titlesec}
\usepackage{url}
\usepackage{xcolor}
\usepackage{wrapfig}
\usepackage{listings}
\usepackage{physics}
\usepackage{cite}

\lstnewenvironment{cpp}[1][]{
\lstset{
upquote=true,
columns=flexible,
basicstyle=\ttfamily,
language=C++,
keywordstyle=\color{blue},
commentstyle=\color{gray},
breaklines,
breakindent=1.5em,
xleftmargin=2em,
xrightmargin=2em,
frame=single,
rulecolor=\color{orange},
backgroundcolor=\color{orange!5},
}}{}

%Draft Watermark
%\usepackage{draftwatermark}
%\SetWatermarkLightness{0.85}
%\SetWatermarkAngle{25}
%\SetWatermarkScale{4}
%\SetWatermarkFontSize{5cm}
%\SetWatermarkText{Draft}

%\usepackage{CJKutf8}
%\begin{CJK}{UTF8}{min}このコマンドで、日本語で入力するの為\end{CJK}

\date{\Large{ \today}}
\author{\Large{Sergio Ricossa}}
\title{\huge{\textsf{Designing the game “Lasers!”}}}

\newcommand{\I}{\text{i}}
\newcommand{\E}{\text{e}}
\newcommand{\PI}{\text{\textpi}}
\newcommand{\HC}{\text{H.C.}}
\newcommand{\qop}{\mathsf}


\begin{document}
\maketitle
\tableofcontents

\section{The Problem}

You are targeted by a laser in a rectangular room with four
mirror-walls. The laser can bounce off the mirrors. You have
up to sixteen point-like obstacles that will block the
laser. Your goal is to find a configuration in which the
laser can no longer hit you.

Both you and the shooter are point-like objects, and the
laser can go through the shooter. If the laser hits a corner
of the room, it bounces right back 180$\deg$.

\section{The Solution}

The idea is to consider the equivalent problem of an
infinite square lattice, made of infinitely many copies of
the original rectangular room, with a target point in each.
The laser is always a straight line from the shooter to any
of the targets on the lattice.

When an obstacle is placed, infinitely many copies are
placed in the lattice. The laser has to avoid all of them.

We will denote by $l$ and $L$ the width and length of the
room, respectively.

$\vb*{x_0}=(x_0,y_0)$, is the position of the shooter and,
$\vb*{x_1}=(x_1,y_1)$, that of the target.

Because of the mirrors, our lattice has to be divided into
four sub-lattices. One of them is made of copies of the
original room.  Another one is made of copies flipped upside
down, yet another one is flipped left to right. The last one
is flipped both left to right and upside down. They all have
periodicity $(2l,2L)$.

These can be encoded by two bits, $s_x, s_y\in\{+1,-1\}$,
which are negative only if the lattice is flipped along the
specified axis.

The positions of the targets are labelled by a choice of
$\vb*{s}=(s_x,s_y)$, as well as two integers,
$\vb*{n}=(n_x,n_y)\in\mathbb{Z}$, that denote the position
in the chosen sub-lattice:
\begin{equation}
    \vb*{x}_1^{\vb*{s},\vb*{n}}=(2ln_x + s_xx_1, 2Ln_y + s_yy_1)
\end{equation}
A valid laser trajectory goes from $\vb*{x}_0$ to any of the
$\vb*{x}_1^{\vb*{s}, \vb*{n}}$. We can write it as,
\begin{equation}
    \vb*{x}_\text{laser}(\vb*{s}, \vb*{n}; t) =
        t\vb*{x}_1^{\vb*{s}, \vb*{n}} + (1-t)\vb*{x}_0.
\end{equation}

Thanks to the peculiar $(2l, 2L)$ periodicity, it is
possible to block every trajectory with $\vb*{s}$ fixed with
only 4 obstacles. In addition, this solution is unique.
Since there are 4 values of $\vb*{s}$, at most $4\times4=16$
obstacles are needed to secure every target on the lattice.\\


We show this now. Two points of the lattice are equivelent
if they are equal to each other $\mod{(2l,2L)}$. This
criterion is sufficient, but not necessary, since it only
checks equivalence on the same sub-lattice.

Taking $\vb*{x}_\text{laser}(\vb*{s},
\vb*{n};t)\mod{(2l,2L)}$, we look for a value of $t$ that
yields values from a finite set of points, independent of
$\vb*{n}$. This is equivalent to asking which values of $t$
are such that $\{2ln_xt\mod{2l}|n_x\in\mathbb{Z}\}$ and
$\{2Ln_yt\mod{2L}|n_y\in\mathbb{Z}\}$ are both finite.

This is only possible if $t$ is a rational number. And the
minimal number of obstacles is obtained when $t=0$ and
$t=1$, but that is obviously not permitted by the problem.
It would amount to placing one obstacle on the shooter, or
the target, and calling it done.

Instead, the next best option is $t=1/2$, for which we get 4
points, depending on whether $n_x$ and $n_y$ are even or
odd.

Two or more of these points could still be equivalent. But
we can just check for all 4 points (16, counting all choices
of $\vb*{s}$) to see if they are the same or different. They
are given by the formula,
\begin{equation}
    \vb*{x}_\text{solution}(\vb*{s},\vb*{s}') =
        \left(
            s'_x\frac{x_0 + s_xx_1}{2} \mod{l},
            s'_y\frac{y_0 + s_yy_1}{2} \mod{L}
        \right),
    \label{eq:solution}
\end{equation}
where we introduced a new set of bits,
$\vb*{s}'=(s'_x,s'_y)$. $s'_x=1$ if $n_x$ is even, and
$s'_x=-1$ if $n_x$ is odd.  The same goes for $s'_y$ and
$n_y$.\\

The way we check that there are no other solutions with 16
obstacles or less is by noticing that $\vb*{s}'$ corresponds
to the sub-lattice where the midpoint $\vb*{x}(\vb*{s},
\vb*{n}; 1/2)$ lies.

Indeed, increasing either $n_x$ or $n_y$ by 2 moves the
midpoint $2l$ or $2L$ ahead on the lattice, which means it
stays in the same sub-lattice. If $n_x$ and $n_y$ are set as
either even or odd, then they can only be changed by a
multiple of two. And setting $\vb*{s}'$ amounts to the same
constraint.

For two or more points to be equivalent on a
\emph{sub-lattice}, it is in fact necessary that they be
congruent $\mod{(2l,2L)}$.  Thus, each of the points in
eq.\ref{eq:solution} is the only one that covers all
laser paths, if the target is on the sub-lattice $\vb*{s}$,
and the midpoint is on the sub-lattice $\vb*{s}'$.\\

Programmatically, we can encode $\vb*{s}$ and $\vb*{s}'$
inside a single 4 bit number. There are also a few (quite
litteral) edge cases that need to be checked:
\begin{cpp}
    std::set<std::pair<int,int>> solution;

    for(unsigned i{0} ; i < 16 ; i++)
    {
        int x{( (i&1 ? 1 : -1) * (x0 + (i&2 ? 1 : -1)*x1 )/2 ) % m_width};
        int y{( (i&4 ? 1 : -1) * (y0 + (i&8 ? 1 : -1)*y1 )/2 ) % m_height};
        if(x < 0) x = m_width + x;
        else if(x == 0) x = i&1 ? 0 : m_width;
        if(y < 0) y = m_height + y;
        else if(y == 0) y = i&4 ? 0 : m_height;
        solution.insert(std::pair<int,int>{x,y});
    }
\end{cpp}

\section{Implementation details}

\end{document}